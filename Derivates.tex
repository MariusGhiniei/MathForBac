\documentclass[11pt]{article}
\usepackage[utf8]{inputenc}  
\usepackage[T1]{fontenc}
\usepackage{amsmath,textcomp,amssymb,geometry,graphicx,enumerate}
\usepackage{algorithm} % Boxes/formatting around algorithms
\usepackage[noend]{algpseudocode} % Algorithms
\usepackage{hyperref}
\usepackage{stmaryrd}
\hypersetup{
    colorlinks=true,
    linkcolor=blue,
    filecolor=magenta,      
    urlcolor=blue,
}

\def\Name{Marius G.}  % Your name
\def\Session{}


\title{Rezolvarea Subiectului III - Rolul derivatelor}
\author{\Name}
\date{}
\pagestyle{myheadings}


\newenvironment{qparts}{\begin{enumerate}[{(}a{)}]}{\end{enumerate}}
\def\endproofmark{$\Box$}
\newenvironment{proof}{\par{\bf Proof}:}{\endproofmark\smallskip}
\newcommand\defi{\stackrel{\mathclap{\normalfont\mbox{def}}}{=}}
\newcommand\conv{\stackrel{\mathclap{\normalfont\mbox{conventie}}}{=}}


\textheight=9in
\textwidth=6.5in
\topmargin=-.75in
\oddsidemargin=0.25in
\evensidemargin=0.25in


\begin{document}
\maketitle

\section*{Cu ce ne ajuta derivatele?}
    \begin{flushleft}
        \qquad Rolul primei derivate a unei functii $f$ ne este utila in a studia \textbf{monotonia} ( 
        crescatoare sau descrescatoare) si punctele de extrem pentru $f$.( adica puncte de maxim si minim)
    \end{flushleft}
    
    \begin{flushleft}
        \qquad Rolul derivatei de ordin 2 ne ajuta sa calculam punctele de \textbf{inflexiune}
        ( daca ele exista) si pentru a determina daca $f$ este \textbf{convexa} sau \textbf{concava}
        pe un interval.
    \end{flushleft}

\section*{Definitia derivatei cu limita}
    \begin{flushleft}
    \qquad Spunem despre o functie $f(x)$ ca este derivabila intr-un punct $x_0$ daca:
        \[ \exists \lim_{x \to x_0} \frac{f(x) - f(x_0)}{x - x_0} = L \]
         $\dagger$ Daca $L = \{-\infty, \infty\}$, atunci spunem ca $f$ are derivata dar nu este derivabila.
        \newline
        $\dagger$ Daca $L \in \mathbf{R}$( adica este finit) atunci $f$ este derivabila in $x_0$ si avem ca:
        \[ \lim_{x \to x_0} \frac{f(x) - f(x_0)}{x - x_0} \stackrel{\textit{def}}{=} f(x_0)^\prime  \]
    \end{flushleft}
    
\section*{Reguli de derivare}
    \begin{flushleft}
        \qquad Fie $f,g: \mathbf{R} \to  \mathbf{R}, f $ si $g$ functii derivabile. Atunci avem:  
        \begin{enumerate}[(a).]
            \item $ (f+g)^\prime = f^\prime + g^\prime$
            \item $(f - g)^\prime = f^\prime - g^\prime$
            \item $ (c \cdot f)^\prime = c\cdot f^\prime$, unde c este o constanta
            \item $ (f\cdot g)^\prime\ = f^\prime \cdot g + f \cdot g^\prime $
            \item $ \left( \frac{f}{g} \right) ^\prime = \frac{f^\prime \cdot g - f \cdot g^\prime}{g^2},
            \, \, g(x) \neq 0 $
        \end{enumerate} 
    \end{flushleft}
\newpage

\section*{Monotonia unei functii}
    \begin{flushleft}
        \qquad Fie $f:I \to \mathbf{R}$, o functie derivabila pe I. Avem:
        \begin{enumerate}[(a).]
            \item   Daca $f(x)^\prime \geq 0$, atunci $f$ este crescatoare (pentru oricare x respecta
            relatia).
            \item   Daca $f(x)^\prime \leq 0$, atunci $f$ este descrescatoare (pentru oricare x respecta 
                    relatia).
            \item   Daca $f(x)^\prime > 0$, atunci $f$ este strict crescatoare (pentru oricare x respecta
            relatia).
            \item    Daca $f(x)^\prime < 0$, atunci $f$ este strict descrescatoare (pentru oricare x
            respecta relatia).
            \item   Daca $f(x)^\prime = 0$, atunci $f$ este constanta(pentru oricare x respecta relatia).
        \end{enumerate}
    \end{flushleft}

\section*{Punct de extrem}
    \begin{flushleft}
        \qquad Spunem ca $x_0$ este punct de extrem( critic), daca respecta relatia $f(x_0)^\prime = 0$.
    \end{flushleft}
    
\section*{Punct de inflexiune}
    \begin{flushleft}
        \qquad Spunem ca $x_0$ este punct de inflexiune, daca respecta relatia $f(x_0)'' = 0$.
    \end{flushleft}

\section*{Tabel de variatie(semn)}
\begin{flushleft}
    \qquad  De cele mai multe ori la Subiectul III, la analiza matematica de clasa a XI - a, trebuie sa  
            facem un tabel. Cum il facem? \\
    $\longrightarrow$ Daca am nevoie de monotonia lui $f(x)$, de puncte de minim/maxim, de aratat 
    inegalitati, trebuie sa fac tabelul de variatie a lui $f(x)^\prime$. \\
    $\longrightarrow$ Daca am nevoie de aratat convexitatea/concavitatea lui $f(x)$ pe un interval / tot
    domeniul de definitie a lui $f$ sau trebuie sa arat ca are puncte de inflexiune( sau sa le si calculez),
    trebuie sa fac tabelul de variatie a lui $f(x)''$.
    
    \subsection*{Pasii de urmat pentru tabelul cu $f(x)^\prime$}
    \begin{enumerate}[1).]
        \item   Calculam derivata de ordin 1. ($f(x)^\prime$)
        \item   Egalam derivata cu 0 ($f(x)^\prime = 0)$ si determinam punctele de extrem, daca ele exista.
        \item   Trasam tabelul si punem pe prima coloana ( $x, f(x)^\prime, f(x)$ ), ne uitam in enunt sa
                vedem pe ce interval este definita functia $f$ si punem cele 2 capete pe primul rand din
                tabel.
        \item   Adaugam punctele de extrem( tot pe primul rand) si sub ele( adica pe randul cu $f(x)^\prime$
                scriem 0.( pentru ca derivata in punctul de extrem face 0).
        \item   Trebuie sa punem semnele, daca derivata noastra este de gradul 1 sau 2 ne putem folosi de 
                tabele de variatie de semn din clasa a IX - a sau luam o valoare din intervalul la care
                trebuie sa atasam un semn si inlocuim in $\mathbf{f(x)^\prime}$, ne intereseaza doar semnul 
                ( adica daca e > 0 avem +, daca e < 0 avem -) nu si valoarea.
        \item   Acum trebuie sa trasam sagetele de monotonie si avem: sageata in sus unde avem + si
                sageata in jos unde avem -.
        \item   Daca avem sageata sus urmata de sageata jos, atunci avem punct de maxim, daca avem sageata
                jos urmata de sageata sus avem punct de minim( valorile pentru punctul de minim si cel de
                maxim o sa se calculeze in \textbf{f(x)}.( x - ul il luam de pe primul rand, o sa observam
                ca este de fapt punctul de extrem calculat la pasul 2).
        \item   Ideal ar fi sa facem si limitele in cele 2 capete sa fim siguri ca nu "explodeaza" functia
                intr-un capat.( Pasul acesta este necesar cand lucram cu valori ale lui $f(x)$, pentru 
                exercitii unde trebuie aplicata monotonia \textbf{nu} este necesar). Daca avem de ex un
                capat care nu
                este infinit vom face limita laterala in acel punct( daca avem de exemplu: $(0, \infty)$
                facem limita la stanga in 0 si limita la $\infty$).
    \end{enumerate}

    \subsubsection*{Interpretarea tabelului cu $f(x)^\prime$}
    \qquad Am terminat tabelul, ce fac acum? \\
    \qquad Vom interpreta tabelul, aici difera de la exercitiu la exercitiu, dar daca citim cu 
            \textbf{atentie} enuntul vom lua punctaj maxim( ne uitam la ultima sectiune pentru diferite 
            exemple).
    \begin{enumerate}[I.]
        \item   Unde avem + in tabel, $f$ este crescatoare. Daca sunt mai multe intervale in care $f$ este
                crescatoare o sa unim intervalele prin reuniune ($\cup$).
        \item   Unde avem - in tabel, $f$ este descrescatoare. Daca sunt mai multe intervale in care $f$
                este descrescatoare o sa unim intervalele prin reuniune ($\cup$).
        \item   Daca avem mai multe puncte de minim/maxim il vom alege pe cel mai mic/mare ca fiind cel
                global( adica peste toata functia).
        
    \end{enumerate}

    \subsection*{Pasii de urmat pentru tabelul cu $f(x)''$}
        \begin{enumerate}[1).]
            \item   Calculam derivata de ordin 1, apoi derivata de ordin 2.
            \item   Egalam derivata de ordin 2 cu 0 ($f(x)'' = 0)$ si determinam punctele de inflexiune, 
                    daca ele exista.
            \item   Trasam tabelul si punem pe prima coloana ( $x, f(x)'', f(x)$ ), ne uitam in enunt
                    sa vedem pe ce interval este definita functia $f$ si punem cele 2 capete pe primul rand
                    din tabel.
            \item   Adaugam punctele de inflexiune( tot pe primul rand) si sub ele( adica pe randul cu  
                    $f(x)''$ scriem 0.( pentru ca derivata in punctul de inflexiune face 0).
            \item   Trebuie sa adaugam semnele derivatei de ordin 2, luam o valoare din intervalul caruia
                    trebuie sa ii atasam un semn si inlocuim in $f(x)''$ si ne intereseaza doar semnul nu
                    si valoarea.
            \item   Daca avem + pe randul cu $f(x)''$, pe randul cu $f(x)$ vom trasa un \textit{smile face} 
                    \\
                    Daca avem - pe randul cu $f(x)''$, pe randul cu $f(x)$ vom trasa un \textit{sad face} 
                    
        \end{enumerate}
    
\end{flushleft}
\newpage
    \subsubsection*{Interpretarea tabelului cu $f(x)''$}
    \begin{enumerate}[I.]
        \item   Unde avem + in tabel( smile face) $f$ este convexa. Daca sunt mai multe intervale in care
                $f$ este convexa o sa le reunim( $\cup$).
        \item   Unde avem - in tabel( sad face) $f$ este concava. Daca sunt mai multe intervale in care $f$
                este concava o sa le reunim( $\cup$).
        \item   Punctele de inflexiune sunt date de $f(x)'' = 0$, daca ne cere valoarea functiei in punctele
                de inflexiune, inlocuim in $f(x)$ cu punctele respective.
    \end{enumerate}
    
\section*{Calculul Asimptotelor}
    \subsection*{Orizontala}
        \begin{flushleft}
        \qquad Fie $f:I \to \mathbf{R}$. Spunem ca dreapta $y = l$ este asimptota orizontala la graficul 
        functiei $f$ spre
        $ \pm \infty$, daca $l \in \mathbf{R}$ (adica daca l exista si este diferit de $\pm \infty$), unde
        \[ l = \lim_{x \to \pm \infty} f(x)\]
        \end{flushleft}
    \subsection*{Oblica}
        \begin{flushleft}
            \qquad Fie $f:I \to \mathbf{R}$. Spunem ca dreapta $y = mx + n$ este asimptota oblica la 
            graficul functiei 
            $f$ spre $\pm \infty$ , daca $m,n \in \mathbf{R}$( adica m si n exista si sunt diferite de $\pm 
            \infty$), unde
            \[ m = \lim_{x \to \pm \infty} \frac{f(x)}{x}, \qquad n = \lim_{x \to \pm \infty} (f(x) - m 
            \cdot x)\]
            \qquad Daca ne da $m = 0$, inseamna ca avem o asimptota orizontala( ar trebuie sa ne mai uitam
            odata peste calcule)
        \end{flushleft}
        
\subsection*{Verticale}
    \begin{flushleft}
        \qquad Fie $f:I \to \mathbf{R}$ si $x_0$ un punct de acumulare, $a \in \mathbf{R}$($\ne \pm
        \infty$).
        \newline
        \qquad Spunem ca dreapta $x = a$ este asimptota verticala la stanga lui $f$, daca: 
        \[\lim_{x \downarrow a } f(x) = \lim_{x \to a, x < a} f(x) = \pm \infty\]
         \qquad Spunem ca dreapta $x = a$ este asimptota verticala la dreapta lui $f$, daca: 
        \[\lim_{x \uparrow a } f(x) = \lim_{x \to a, x > a} f(x) = \pm \infty\] 
        \qquad Daca avem asimptota verticala la stanga si dreapta atunci, spunem ca dreapta $x = a$ este 
        asimptota verticala la graficului lui $f$.
    \end{flushleft}
    
\subsection*{Ecuatia tangentei la grafic}
    \begin{flushleft}
        \qquad Ecuatia tangentei la graficul lui $f$ intr-un punct de abscisa $x_0$ are formula:
        \[y - f(x_0) = f(x_0)^\prime(x - x_0)\].
    \end{flushleft}
    
\section*{Formule de calcul si anumite observatii}
    \begin{flushleft}
        $(a - b)\cdot (a + b) = a^2 - b^2$ \newline
        $(a+b)^2 = a^2 + 2\cdot a\cdot b + b^2$ \newline
        $(a-b)^2 = a^2 - 2\cdot a\cdot b + b^2$ \newline
        $ln(e) = 1$ , $ln(1) = ln(e^0) = 0$ , $ln(0) = - \infty (conventie)$ \newline
        $arctg(0) = 0$ , $arctg(1) = \frac{\pi}{4}$ \newline
        Ecuatia $e^x = 0$ \textbf{NU} are solutii in $\mathbf{R}$. \newline
        Ecuatia $e^x = 1$ are unica solutie $x = 0$ \newline
        $\lim\limits_{x \to \pm \infty} \frac{1}{x} = \frac{1}{\pm \infty} = 0$ \newline
        $\lim\limits_{x \to \infty} e^x = \infty$ \textbf{SI}  $\lim\limits_{x \to -\infty} e^x = 0$
    \end{flushleft}

\section*{Idei uzuale pentru c - uri}
\begin{enumerate}[1)]
    \item   Demonstrati ca $a \leq f(x) \leq b, \quad \forall x \in [x_0, x_1]$
        \item[] Facem tabelul de variatie si aratam ca a si b sunt efectiv valorile pentru punctele de
                minim si maxim( cand este un interval mare care contine macar intr-o parte $\infty$)
                (aici este necesar sa facem si limitele in capete ca sa spunem ca punctele gasite sun
                globale).\\
                Sau daca avem un interval mai mic aratam ca pe intervalul $[x_0, x_1]$ functia $f$ este
                monotona( crescatoare sau descrescatoare) si ca $f(x_0) = a, f(x_1) = b$, de unde iese
                concluzia. 

    \item   Aratati ca $f(x) \geq \frac{1}{e}, \quad \forall x \in [x_0, x_1]$ 
        \item[] Facem tabelul si aratam ca punctul de minim pe $[x_0, x_1]$ are valoarea $\frac{1}{e}$.
        
    \item   Aratati ca $f(x) \leq 23, \quad \forall x \in [x_0, x_1]$ 
        \item[] Facem tabelul si aratam ca punctul de maxim pe $[x_0, x_1]$ are valoarea \textit{23}.

    \item   Aratati ca $f(x) \geq x^2 - 3, \quad \forall x \in [x_0,x_1]$
        \item[] Ducem ce e dupa "$\geq$" in partea stanga, adica $f(x) - x^2 +3 \geq 0$. \\
                Notam cu $g(x) = f(x) - x^2 +3$, facem tabelul de variatie si aratam ca $g(x)^\prime \geq
                0$, \, $\forall x \in [x_0,x_1]$.
                
    \item   Aratati ca $f(x) \leq e^{2x} - 2\cdot \pi, \quad \forall x \in [x_0,x_1]$
        \item[] Ducem ce e dupa "$\leq$" in partea stanga, adica $f(x) - e^{2x} + 2\cdot \pi.$ \\
                Notam cu $g(x) = f(x) -  e^{2x} + \, 2\cdot \pi$, facem tabelul de variatie si aratam ca 
                $g(x)^\prime \leq 0$, \,$\forall x \in [x_0,x_1]$.

    \item   Determinati multimea de valori ale functiei $f$.
        \item[] Tabel de variatie + limite in capete si multimea de valori este de fapt "pe unde se plimba
                $f$ -ul " cu ajutorul punctelor de minim,minim( ar trebui sa obtinem puncte globale).
                
    \item   Determinati intervalele de monotonie ale functiei $f$.
        \item[] Facem tabel si acolo unde apare + inseamna ca $f(x)$ este crescatoare, si acolo unde apare
                - inseamna ca $f(x)$ este descrescatoare.( daca sunt mai multe intervale cu + sau - vom
                reuni ( $\cup$) intervalele). \textbf{Important!} este sa mentionam in scris pe ce 
                interval e crescatoare/descrecatoare, adica: \underline{$f$ este crescatoare} pe $[-12, 24]$
                si \underline{$f$ este descrescatoare} pe $(-\infty,-12) \cup (24, \infty)$.
                
    \item   Aratati ca $f$ e convexa pe $\mathbf{R}$. 
        \item[] Daca trebuie sa aratam ca $f$ e convexa/concava pe tot domeniul de definitie, ne asteptam
                ca $f$ sa nu aiba puncte de inflexiune, adica $f(x)'' = 0$ nu are solutii reale( sau avem
                derivate de genul $\frac{23}{x^2 - 3x}$ care nu are puncte de inflexiune).
   \item    Aratati ca $a \leq f(x) + f(y) \leq b$, pentru $\forall x,y \in \mathbf{R} $.
        \item[] Vom face tabelul de variatie si ne asteptam ca $\frac{a}{2}$ sa fie punct de minim si ca
                $\frac{b}{2}$ sa fie punct de maxim daca $f$ este crescatoare si invers daca $f$ este
                descrescatoare.\\
                Vom scrie ca: 
                \begin{gather*}
                    f(x) \: \text{pe} \: [x_0,x_1] \: \text{este crescatoare atunci:}  \\
                    \Rightarrow \frac{a}{2} \leq f(x) \leq \frac{b}{2}, \forall x \in \mathbf{R} \\
                    \Rightarrow \frac{a}{2} \leq f(y) \leq \frac{b}{2}, \forall y \in \mathbf{R} \\
                    \Longrightarrow  \frac{a}{2} +  \frac{a}{2} \leq f(x) + f(y) \leq \frac{b}{2} + \frac{b}
                    {2},
                    \forall x,y \in \mathbf{R} \\
                    \text{adica, } \: a \leq f(x) + f(y) \leq b, \forall x,y \in \mathbf{R}, \text{de unde
                    reiese si concluzia problemei.}
                \end{gather*}
                Analog si pentru cand $f(x)$ este descrescatoare pe $[x_0, x_1]$.
                
\end{enumerate}

\end{document}

